\documentclass[a4paper]{article}

\setlength{\parskip}{0.1em}
\newcommand{\tab}{~ \qquad}
\input{Algo1Macros}
\usepackage{caratula} % Version modificada para usar las macros de algo1 de ~> https://github.com/bcardiff/dc-tex

\begin{document}

\titulo{Mini-TP de Especificaci\'on}
\subtitulo{Ejercicios de \LaTeX}
\fecha{}
\materia{Algoritmos y Estructuras de Datos I}
\grupo{Grupo XX}

% Pongan cuantos integrantes quieran
\integrante{Bond, James}{007}{bond@mi6.co.uk}
%\integrante{Apellido, Nombre2}{002/01}{email2@dominio.com}
%\integrante{Apellido, Nombre3}{003/01}{email3@dominio.com}
%\integrante{Apellido, Nombre4}{004/01}{email4@dominio.com}

\maketitle

\section{Ejemplos}

\begin{itemize}
\item Dados dos números devolver el resto de su división

\begin{proc}{cociente}{\In a,b: \ent, \Out result: \ent}{}
    \pre{b \neq 0}
    \post{result = a \ mod \ b}
\end{proc}

\item Un predicado: \\
\pred{noRepe}{l:\TLista{\ent}}{\\
\tab (\forall i:\ent)( 0 \leq i < l \implicaLuego l[i] \notin subseq(l,0,i))}

\item Un auxiliar: \\
\aux{primosMenores}{n:\ent}{\ent}{\sum_{i=2}^{n-1}\IfThenElse{soy\_primo(i)}{1}{0}}

\item Una f\'ormula: \\
$((p \y (q \lor r)) \Iff ((p \y q) \lor (p \y r))$

 
\end{itemize}

\section{Problemas}

\begin{enumerate}


\item Sea $m$ una lista de lista de tipo $\ent$, escribir el auxiliar acumPares tal que retorne la suma de todas las posiciones impares de cada secuencia.

\item Sea $m$ una lista de lista de tipo $\ent$ y un entero N hacer un predicado que devuelva verdadero si alguna de las listas tiene longitud N y ademas contiene el elemento N.


\item Escribir en \LaTeX  la expresion de los procedimientos siguientes.

\begin{itemize}
\item  \textbf{proc mayorPrimoQueDivide(in x:$\ent$, in y:$\ent$, out res: $\bool$)}: que sea verdadero si $y$ es el mayor primo que divide a $x$.

\item \textbf{proc esMatrizIdentidad(in m: $\TLista{\TLista{\ent}}$, out res: $\bool$)}: que indica por verdadero o falso si una matriz cuadrada v\'alida es identidad.

\end{itemize}
\end{enumerate}




\section{Decisiones tomadas}

\end{document}
